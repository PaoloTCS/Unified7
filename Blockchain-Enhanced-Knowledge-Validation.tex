\section{Blockchain-Enhanced Knowledge Validation}

In this section, we integrate blockchain concepts with the Nibbler Algorithm and the Fundamental Interaction Language (FIL) to create a decentralized knowledge validation mechanism. This method ensures the integrity and consistency of knowledge propagation by utilizing blockchain-inspired features such as smart contracts, consensus, and a dynamically advancing truth wall.

\subsection{Smart Contracts for Knowledge Validation}
Smart contracts play a central role in validating new knowledge blocks before they are incorporated into the knowledge graph. These contracts define the rules for verification and consensus, ensuring that new information meets predefined criteria for acceptance.

\begin{definition}[Knowledge Block]
A Knowledge Block $KB$ is a tuple $(D, H_{\text{prev}}, P, V)$ where:
\begin{itemize}
    \item $D$ is the data or knowledge content.
    \item $H_{\text{prev}}$ is the hash of the previous block.
    \item $P$ is the proof (either a formal proof or empirical validation).
    \item $V$ is a set of verification signatures from surrounding nodes.
\end{itemize}
\end{definition}

Each node in the system validates a new Knowledge Block by checking whether it adheres to the conditions laid out in a smart contract. The contract ensures that the block's hash matches the proof and that a sufficient number of verifications have been provided.

\begin{definition}[Smart Contract Category]
The category $\mathcal{S}$ of smart contracts consists of:
\begin{itemize}
    \item Objects: Contract states $S = \{s_1, s_2, ..., s_n\}$.
    \item Morphisms: Transitions between contract states $f_{ij}: s_i \rightarrow s_j$.
    \item Composition: For $f_{ij}: s_i \rightarrow s_j$ and $f_{jk}: s_j \rightarrow s_k$, composition $f_{jk} \circ f_{ij}: s_i \rightarrow s_k$.
\end{itemize}
\end{definition}

Smart contracts thus model the verification process as a series of transitions between contract states, ensuring that the new knowledge block has been properly validated.

\subsection{Smart Contract-FIL Correspondence}
To formalize the interaction between smart contracts and the Fundamental Interaction Language (FIL), we propose the following theorem:

\begin{theorem}[Smart Contract-FIL Correspondence]
There exists a functor $F: \mathcal{S} \rightarrow \mathcal{C}_{FIL}$ that maps:
\begin{itemize}
    \item Contract states to FIL entities (such as knowledge blocks).
    \item Contract transitions to FIL interactions (such as verification or consensus processes).
\end{itemize}
\end{theorem}

This functor ensures that the knowledge blocks and the FIL entities interact via smart contracts, thereby enforcing verification conditions before the knowledge is accepted into the knowledge graph.

\subsection{Blockchain-Nibbler Algorithm}
The Nibbler Algorithm is extended with blockchain-like consensus mechanisms to validate new knowledge blocks before they are added to the graph. Each node in the network participates in the validation process by verifying incoming knowledge blocks and propagating the verified blocks further.


% \caption{Blockchain-Nibbler Algorithm}
\begin{algorithmic}[1]
   \State For each node $n$ in the network:
   \State Propagate information as in standard Nibbler.
   \If{new knowledge $k$ is generated}
       \State Create a potential Knowledge Block $KB(k)$.
       \State Initiate consensus process with surrounding nodes.
       \If{consensus is reached}
           \State Update the Truth Wall: $TW = TW \cup \{KB(k)\}$.
       \EndIf
   \EndIf
   \State Continue propagation from nodes with updated Truth Walls.
\end{algorithmic}


The blockchain-inspired approach ensures that the validation process is decentralized. Each node acts as a verifier, and only when consensus is reached does the new block become part of the truth wall.

\subsection{The Truth Wall}
The Truth Wall represents the cumulative knowledge that has passed through verification. As more knowledge blocks are validated and reach consensus, the truth wall advances, incorporating only the blocks that have been verified by a sufficient number of surrounding nodes.

\begin{definition}[Truth Wall]
The Truth Wall $TW(t)$ at time $t$ is defined as:
\begin{equation}
    TW(t) = \{KB : \text{consensus}(KB) = \text{true}, \text{time}(KB) \leq t\}
\end{equation}
where $\text{consensus}(KB)$ checks whether the Knowledge Block has reached consensus, and $\text{time}(KB)$ is the timestamp of the block.
\end{definition}

As the truth wall advances, it ensures that only verified and trusted knowledge blocks are incorporated into the system, preventing false or incomplete information from spreading.

\subsection{Knowledge Integrity Theorem}
The blockchain-inspired validation process ensures that the likelihood of false information being accepted into the system decreases exponentially as more nodes participate in the validation.

\begin{theorem}[Knowledge Integrity]
Given a Knowledge Graph $G$ with blockchain verification, the probability of accepting false information $P(\text{false})$ decreases exponentially with the number of verification nodes:
\begin{equation}
    P(\text{false}) \leq e^{-\alpha N}
\end{equation}
where $N$ is the number of nodes participating in the verification process, and $\alpha$ is a system-specific constant.
\end{theorem}

This theorem highlights the security and robustness of the blockchain-enhanced validation process. As more nodes participate, the system becomes increasingly resilient against the introduction of false knowledge.

\subsection{Decentralization and Trust}
By incorporating blockchain-like consensus mechanisms, this approach decentralizes the validation process, ensuring that no single node or entity has control over what knowledge is accepted. This decentralization not only enhances trust but also makes the system more resistant to tampering, bias, or errors.

The smart contracts, combined with the truth wall and consensus mechanisms, maintain the integrity of the knowledge graph. Knowledge propagation is governed by the same rules across all nodes, ensuring a consistent and verifiable knowledge base.

\section{Conclusion}

In this section, we have described a method for enhancing the Nibbler Algorithm and FIL with blockchain-inspired decentralized knowledge validation. The use of smart contracts, consensus, and the truth wall ensures that only verified knowledge is propagated throughout the system. The Knowledge Integrity Theorem further formalizes the robustness of the system, showing how the probability of accepting false knowledge decreases as the number of verifying nodes increases.
